\documentclass[journal]{IEEEtran}
\usepackage[a5paper, margin=10mm, onecolumn]{geometry}
%\usepackage{lmodern} % Ensure lmodern is loaded for pdflatex
\usepackage{tfrupee} % Include tfrupee package

\setlength{\headheight}{1cm} % Set the height of the header box
\setlength{\headsep}{0mm}     % Set the distance between the header box and the top of the text

\usepackage{gvv-book}
\usepackage{gvv}
\usepackage{cite}
\usepackage{amsmath,amssymb,amsfonts,amsthm}
\usepackage{algorithmic}
\usepackage{graphicx}
\usepackage{textcomp}
\usepackage{xcolor}
\usepackage{txfonts}
\usepackage{listings}
\usepackage{enumitem}
\usepackage{mathtools}
\usepackage{gensymb}
\usepackage{comment}
\usepackage[breaklinks=true]{hyperref}
\usepackage{tkz-euclide} 
\usepackage{listings}
% \usepackage{gvv}                                        
\def\inputGnumericTable{}                                 
\usepackage[latin1]{inputenc}                                
\usepackage{color}                                            
\usepackage{array}                                            
\usepackage{longtable}                                       
\usepackage{calc}                                             
\usepackage{multirow}                                         
\usepackage{hhline}                                           
\usepackage{ifthen}                                           
\usepackage{lscape}
\begin{document}
\bibliographystyle{IEEEtran}
\title{2011-PH-'1-13'}
\author{EE24BTECH11009-Mokshith}
{\let\newpage\relax\maketitle}
\renewcommand{\thefigure}{\theenumi}
\renewcommand{\thetable}{\theenumi}
\setlength{\intextsep}{10pt} % Space between text and floats
\numberwithin{equation}{enumi}
\numberwithin{figure}{enumi}
\renewcommand{\thetable}{\theenumi
}
\begin{enumerate}
\item Two matrices $A$ and $B$ are said to be similar if $B = P^{-1}AP$ for some invertible matrix $P$. Which of the following statements is NOT TRUE?
\begin{enumerate}
    \item Det $A$ = Det $B$
    \item Trace of $A$ = Trace of $B$
    \item $A$ and $B$ have the same eigenvectors
    \item $A$ and $B$ have the same eigenvalues
    \end{enumerate}
\item If a force $F$ is derivable from a potential function $V(r)$, where $r$ is the distance from the origin of the coordinate system, it follows that
\begin{enumerate}
    \item $\overrightarrow{\nabla} \times \overrightarrow{F} = 0$
    \item $\overrightarrow{\nabla} \cdot \overrightarrow{F} = 0$
    \item $\overrightarrow{\nabla} V = 0$
    \item $\nabla^2 V = 0$
\end{enumerate}
\item The quantum mechanical operator for the momentum of a particle moving in one dimension is given by
\begin{enumerate}
    \item $i\hbar\frac{d}{dx}$
    \item $-i\hbar\frac{d}{dx}$
    \item $i\hbar\frac{\partial}{\partial t}$
    \item $-\frac{\hbar^2}{2m}\frac{d^2}{dx^2}$
\end{enumerate}
\item A Carnot cycle operates on a working substance between two reservoirs at temperatures $T_1$ and $T_2$, with $T_1 > T_2$. During each cycle, an amount of heat $Q_1$ is extracted from the reservoir at $T_1$ and an amount $Q_2$ is delivered to the reservoir at $T_2$. Which of the following statements is INCORRECT?
\begin{enumerate}
    \item work done in one cycle is $Q_1 - Q_2$
    \item $\frac{Q_1}{T_1} = \frac{Q_2}{T_2}$
    \item entropy of the hotter reservoir decreases
    \item entropy of the universe (consisting of the working substance and the two reservoirs) increases
\end{enumerate}
\item In a first order phase transition, at the transition temperature, specific heat of the system
\begin{enumerate}
    \item diverges and its entropy remains the same
    \item diverges and its entropy has finite discontinuity
    \item remains unchanged and its entropy has finite discontinuity
    \item has finite discontinuity and its entropy diverges
    \end{enumerate}
\item The semi-empirical mass formula for the binding energy of nucleus contains a surface correction term. This term depends on the mass number $A$ of the nucleus as
\begin{enumerate}
    \item $A^{-1/3}$
    \item $A^{1/3}$
    \item $A^{2/3}$
    \item $A$
\end{enumerate}
\item The population inversion in a two level laser material CANNOT be achieved by optical pumping because
\begin{enumerate}
    \item the rate of upward transitions is equal to the rate of downward transitions
    \item the upward transitions are forbidden but downward transitions are allowed
    \item the upward transitions are allowed but downward transitions are forbidden
    \item the spontaneous decay rate of the higher level is very low
\end{enumerate}
\item The temperature ($T$) dependence of magnetic susceptibility ($\chi$) of a ferromagnetic substance with a Curie temperature ($T_c$) is given by
\begin{enumerate}
    \item $\frac{C}{T-T_c}$, for $T < T_c$
    \item $\frac{C}{T-T_c}$, for $T > T_c$
    \item $\frac{C}{T+T_c}$, for $T > T_c$
    \item $\frac{C}{T+T_c}$, for all temperatures
\end{enumerate}
where $C$ is constant.
\item The order of magnitude of the energy gap of a typical superconductor is
\begin{enumerate}
    \item 1 MeV
    \item 1 KeV
    \item 1 eV
    \item 1 meV
\end{enumerate}
\item Which of the following statements is CORRECT for a common emitter amplifier circuit?
\begin{enumerate}
    \item The output is taken from the emitter
    \item There is 180$^\circ$ phase shift between input and output voltages
    \item There is no phase shift between input and output voltages
    \item Both $p$-$n$ junctions are forward biased
\end{enumerate}
\item A $3\times3$ matrix has elements such that its trace is 11 and its determinant is 36. The eigenvalues of the matrix are all known to be positive integers. The largest eigenvalue of the matrix is
\begin{enumerate}
    \item 18
    \item 12
    \item 9
    \item 6
\end{enumerate}
\item A heavy symmetrical top is rotating about its own axis of symmetry (the $z$-axis). If $I_1$, $I_2$ and $I_3$ are the
principal moments of inertia along $x$, $y$ and $z$ axes respectively, then
\begin{enumerate}
    \item $I_2 = I_3$; $I_1 \neq I_2$
    \item $I_1 = I_3$; $I_1 \neq I_2$
    \item $I_1 = I_2$; $I_1 \neq I_3$
    \item $I_1 \neq I_2 \neq I_3$
\end{enumerate}
\item An electron with energy $E$ is incident from left on a potential barrier, given by
\[V(x) = \begin{cases} 
    0 & \text{for } x < 0 \\
    V_0 & \text{for } x > 0
\end{cases}\]
as shown in the figure.
\begin{figure}[H]
\centering
\resizebox{5cm}{!}{
\begin{circuitikz}
    \draw[->] (-2,0) -- (4,0) node[right] {$x$};
    \draw[->] (0,-0.5) -- (0,3) node[above] {$V(x)$};
    \draw[thick] (-2,0) -- (0,0);
    \draw[thick] (0,2) -- (4,2);
    \node at (-0.25,2) {$V_0$};
    \draw[dashed,->] (-1.5,1.5) -- (0,1.5) node[right] {$E$};
    \node[below] at (0.25,0) {0};
\end{circuitikz}
}
\label{fig:my_label}
\end{figure}
For $E < V_0$, the space part of the wave function for $x > 0$ is of the form
\begin{enumerate}
    \item $e^{\alpha x}$
    \item $e^{-\alpha x}$
    \item $e^{i\alpha x}$
    \item $e^{-i\alpha x}$
\end{enumerate}
where $\alpha$ is a real positive quantity.
\end{enumerate}
\end{document}
