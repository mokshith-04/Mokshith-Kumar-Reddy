\documentclass[journal]{IEEEtran}
\usepackage[a5paper, margin=10mm, onecolumn]{geometry}
%\usepackage{lmodern} % Ensure lmodern is loaded for pdflatex
\usepackage{tfrupee} % Include tfrupee package

\setlength{\headheight}{1cm} % Set the height of the header box
\setlength{\headsep}{0mm}     % Set the distance between the header box and the top of the text

\usepackage{gvv-book}
\usepackage{gvv}
\usepackage{cite}
\usepackage{amsmath,amssymb,amsfonts,amsthm}
\usepackage{algorithmic}
\usepackage{graphicx}
\usepackage{textcomp}
\usepackage{xcolor}
\usepackage{txfonts}
\usepackage{listings}
\usepackage{enumitem}
\usepackage{mathtools}
\usepackage{gensymb}
\usepackage{comment}
\usepackage[breaklinks=true]{hyperref}
\usepackage{tkz-euclide} 
\usepackage{listings}
% \usepackage{gvv}                                        
\def\inputGnumericTable{}                                 
\usepackage[latin1]{inputenc}                                
\usepackage{color}                                            
\usepackage{array}                                            
\usepackage{longtable}                                       
\usepackage{calc}                                             
\usepackage{multirow}                                         
\usepackage{hhline}                                           
\usepackage{ifthen}                                           
\usepackage{lscape}
\usetikzlibrary{patterns}
\begin{document}
\bibliographystyle{IEEEtran}
\title{2021-XE-'14-26'}
\author{EE24BTECH11009-Mokshith}
{\let\newpage\relax\maketitle}
\renewcommand{\thefigure}{\theenumi}
\renewcommand{\thetable}{\theenumi}
\setlength{\intextsep}{10pt} % Space between text and floats
\numberwithin{equation}{enumi}
\numberwithin{figure}{enumi}
\renewcommand{\thetable}{\theenumi
}
\begin{enumerate}[start=4]
\item Let $f\brak{x}$ be a non-negative continuous function of real variable $x$. If the area under the curve $y = f\brak{x}$ from $x = 0$ to $x = a$ is $\frac{a^2}{2} + \frac{\pi}{2}\sin a + \frac{\pi}{2}\cos a - \frac{\pi}{2}$, then the value of $f\brak{\frac{\pi}{2}}$ is \underline{\hspace{1cm}}\brak{\text{round off to one decimal place}}.

\item If the numerical approximation of the value of the integral $\int_0^4 2^{\alpha x} dx$ using the Trapezoidal rule with two sub intervals is 9, then the value of the real constant $\alpha$ is \underline{\hspace{1cm}}\brak{\text{round off to one decimal place}}.
\item Let the transformation $y\brak{x} = e^x v\brak{x}$ reduce the ordinary differential equation
$$x\frac{d^2y}{dx^2} + 2\brak{1-x}\frac{dy}{dx} + \brak{x-2}y = 0; \quad x > 0$$
to
$$\alpha x\frac{d^2v}{dx^2} + \beta\frac{dv}{dx} + \gamma v = 0,$$
where $\alpha$, $\beta$, $\gamma$ are real constants. Then, the arithmetic mean of $\alpha$, $\beta$, $\gamma$ is \underline{\hspace{1cm}}\brak{\text{round off to three decimal places}}.
\item A person, who speaks the truth 3 out of 4 times, throws a fair dice with six faces and informs that the outcome is 5. The probability that the outcome is really 5 \underline{\hspace{1cm}}\brak{\text{round off to three decimal places}}.
\item Let $f\brak{x,y} = x^4 + y^4 - 2x^2 + 4xy - 2y^2 + \alpha$ be a real valued function. Then, which one of the following statements is TRUE for all $\alpha$?
\begin{enumerate}
    \item $\brak{0,0}$ is not a stationary point of $f$
    \item $f$ has a local maxima at $\brak{0,0}$
    \item $f$ has a local minima at $\brak{0,0}$
    \item $f$ has a saddle point at $\brak{0,0}$
\end{enumerate}
\item Let $u\brak{x,y} = \brak{x^2 - y^2}v\brak{x,y}$ be such that both $u\brak{x,y}$ and $v\brak{x,y}$ satisfy the Laplace equation in a domain $\Omega$ of the $xy$-plane. Then, which one of the following is TRUE in $\Omega$?
\begin{enumerate}
    \item $x\frac{\partial v}{\partial x} - y\frac{\partial v}{\partial y} = 0$
    \item $x\frac{\partial v}{\partial x} + y\frac{\partial v}{\partial y} = 0$
    \item $x\frac{\partial v}{\partial y} - y\frac{\partial v}{\partial x} = 0$
    \item $x\frac{\partial v}{\partial y} + y\frac{\partial v}{\partial x} = 0$
\end{enumerate}
\item Let $I$ denote the identity matrix of order 7, and $A$ be a $7 \times 7$ real matrix having characteristic polynomial $C_A\brak{\lambda} = \lambda^2\brak{\lambda - 1}^\alpha\brak{\lambda + 2}^\beta$, where $\alpha$ and $\beta$ are positive integers. If $A$ is diagonalizable and $rank\brak{A} = rank\brak{A + 2I}$, then $rank\brak{A - I}$ is \underline{\hspace{1cm}}\brak{\text{in integer}}.
\item Let $C_1$ be the line segment from $\brak{0,1}$ to $\brak{\frac{4}{5}, \frac{3}{5}}$, and let $C_2$ be the arc of the circle $x^2 + y^2 = 1$ from $\brak{0,1}$ to $\brak{\frac{4}{5}, \frac{3}{5}}$. If
$$\alpha = \int_{C_1} \brak{\frac{2x}{y}i + \frac{1-x^2}{y^2}j} \cdot dr \text{ and } \beta = \int_{C_2} \brak{\frac{2x}{y}i + \frac{1-x^2}{y^2}j} \cdot dr,$$
where $r = xi + yj$, then the value of $\alpha^2 + \beta^2$ is \underline{\hspace{1cm}}\brak{\text{round off to two decimal places}}.
\end{enumerate}
\section*{Fluid Mechanics (XE-B)}
\begin{enumerate}[start=1]
\item The general relationship between shear stress, $\tau$, and the velocity gradient $\brak{du/dy}$ for a fluid is given by $\tau = k\brak{du/dy}^n$, where $k$ is a constant with appropriate units. The fluid is Newtonian if
\begin{enumerate}
    \item $n > 1$
    \item $n < 1$
    \item $n = 1$
    \item $n = 0$
\end{enumerate}
\item Which one of the following options is TRUE?
\begin{enumerate}
    \item Pathlines and streaklines are the same in an unsteady flow, and streamlines are tangential to the local fluid velocity at a point.
    \item Streamlines are perpendicular to the local fluid velocity at a point, and streamlines and streaklines are the same in a steady flow.
    \item Pathlines and streaklines are the same in an unsteady flow, and streamlines and streaklines are the same in a steady flow.
    \item Streamlines are tangential to the local fluid velocity at a point, and streamlines and streaklines are the same in a steady flow.
    \end{enumerate}
\item If $P_m = 1.2 Pa$ and $P_{out} =1.0 Pa$ are the average pressures at inlet and outlet
respectively for a fully-developed flow inside a channel having a height of $50 cm$, then the absolute value of average shear stress $\brak{in Pa}$ acting on the walls of the channel of length 5 m is
\begin{enumerate}
    \item 0.005
    \item 0.02
    \item 0.01
    \item 0.05
\end{enumerate}
\item Consider the fully-developed flow of a Newtonian fluid \brak{\text{density $\rho$; viscosity $\mu$}} through a smooth pipe of diameter $D$ and length $L$. The average velocity of the flow is $V$. If the length of the pipe is doubled, keeping $V,D,\rho,\mu$ constant, the friction factor
\begin{enumerate}
   \item increases by two times
   \item remains the same  
   \item decreases by two times
   \item increases by four times
\end{enumerate}
\item The absolute value of pressure difference between the inside and outside of a spherical soap bubble of radius, $R$, and surface tension, $\gamma$, is:
\begin{enumerate}
   \item $\frac{2\gamma}{R}$
   \item $\frac{\gamma}{R}$
   \item $\frac{\gamma}{2R}$
   \item $\frac{4\gamma}{R}$
\end{enumerate}
\end{enumerate}
\end{document}
