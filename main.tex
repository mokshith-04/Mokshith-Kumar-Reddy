\let\negmedspace\undefined
\let\negthickspace\undefined
\documentclass[journal,12pt,twocolumn]{IEEEtran}
\usepackage{cite}
\usepackage{amsmath,amssymb,amsfonts,amsthm}
\usepackage{algorithmic}
\usepackage{graphicx}
\usepackage{textcomp}
\usepackage{xcolor}
\usepackage{txfonts}
\usepackage{listings}
\usepackage{enumitem}
\usepackage{mathtools}
\usepackage{gensymb}
\usepackage{comment}
\usepackage[breaklinks=true]{hyperref}
\usepackage{tkz-euclide} 
\usepackage{listings}
\usepackage{gvv}                                        
\def\inputGnumericTable{}                                 
\usepackage[latin1]{inputenc}                                
\usepackage{color}                                            
\usepackage{array}                                            
\usepackage{longtable}                                       
\usepackage{calc}                                             
\usepackage{multirow}                                         
\usepackage{hhline}                                           
\usepackage{ifthen}                                           
\usepackage{lscape}

\newtheorem{theorem}{Theorem}[section]
\newtheorem{problem}{Problem}
\newtheorem{proposition}{Proposition}[section]
\newtheorem{lemma}{Lemma}[section]
\newtheorem{corollary}[theorem]{Corollary}
\newtheorem{example}{Example}[section]
\newtheorem{definition}[problem]{Definition}
\newcommand{\BEQA}{\begin{eqnarray}}
\newcommand{\EEQA}{\end{eqnarray}}
\newcommand{\define}{\stackrel{\triangle}{=}}
\theoremstyle{remark}
\newtheorem{rem}{Remark}
\begin{document}

\bibliographystyle{IEEEtran}
\vspace{3cm}

\title{ASSINGMENT-01}
\author{EE24BTECH11009 - Mokshith Kumar Reddy$^{*}$% <-this % stops a space
}
\maketitle
\newpage
\bigskip
\begin{enumerate}
\item[3.] The value of the expression $^{47}C_4 + ^5_{j=1}\sum ^{52-j}C_3$ is equal to\hfill{$(1982-2$ Marks)}\\
(a) $^{47}C_5$\quad\quad(b) $^{52}C_5$\\
(c)$^{52}C_4$\quad\quad(d) none of these\\
\item[4.]Eight chairs are numbered 1 to 8. Two women and three men wish to occupy one chair each.Frist the women choose the chairs from amongst the chair marked 1to4;andthen the men select the chairs from amongst the remaining. the number of possible arrangements is
\hfill{(1982-2 Marks)}\\
(a) $^6C_3\times^4C_2$\quad\quad(b)$^4P_2\times^4P_3$\\
(c) $^4C_2+^4P_3$\quad\quad(d) none of these\\[4pt]
\item[5.]A five-digit numbers divisible by 3 is to be formed using the numeral 0,1,2,3,4 and 5, without repetition. the total number of ways this can be done is
\hfill{(1989-2 Marks)}\\
(a) $216$\quad(b) $240$\quad(c) 600\quad(d) 3125\\[4pt]
\item[6.]How many different nine digit numbers can be formed from the number 223355888 by rearranging its digits so that the odd digits occupy even positions ?
\hfill{(2000S)}\\
(a) 16\quad(b) 36\quad(c) 60\quad(d) 4
\item[7.]Let $T_n$ denote the number of triangles which can be formed using the vertices of aregular polygon of $n$ sides. If $T_{n+1}-T_n=21$, then n eqals
\hfill{(2001S)}\\
(a) 5\quad(b) 7\quad(c) 6\quad(d) 4
\item[8.]The number of arrangments of the letters of the word BANANA in which the two N's do not appear adjacently is\\
\hfill{(2002S)}
(a) 40\quad\brak{b} 60\quad(c) 80\quad(d) 100
\item[9.]A rectangle with sides of length $(2m-1)$ and $(2n-1)$ units is divided into squares of unit length by drawing parallel lines as shown in the diagram, then the number of rectangles possible with odd side lengths is
\hfill{($2005S$)}\\
\begin{center}
\begin{tabular}{|c|c|c|c|}
\hline
\quad&\quad&\quad&\quad\\
\hline
\quad&\quad&\quad&\quad\\
\hline
\quad&\quad&\quad&\quad\\
\hline\quad&\quad&\quad&\quad\\
\hline\quad&\quad&\quad&\quad\\
\hline\quad&\quad&\quad&\quad\\
\hline
\end{tabular}
\end{center}
(a) $(m+n-1)^2$\quad(b) $4^{m+n-1}$\\
(c) $m^2n^2$\quad(d) $m(m+1)n(n+1)$
\item[10.] If the LCM of$p,q$ is $r^2t^4s^2$ where $r,s,t$ are prime numbers and $p,q$ are the positive integers then the number of ordered pair$(p,q)$ is \hfill{($2006$)\\
(a) 252\quad(b) 254\quad(c) 225\quad(d) 224
\item[11.]The letters of the word COCHIN are permuted and all the permutations are arranged in an alphabetical order as in an English dictionary. The number of words that appear before the word COCHIN is
\hfill{($2007-3 marks$)}\\
(a) 360\quad(b) 192\quad(c) 96\quad(d) 48
\item[12.]The number of seven digit integers, with sum of the digits equal to 10 and formed by using the digits 1,2,and3 only, is
\hfill{($2009$)}\\
(a) 55\quad(b) 66\quad(c) 77\quad(d) 88
\item[13.]The total number of ways in which 5 balls of different colours can be distributed among 3 persons so that each person gets at least one ball is
\hfill{($2012$)}\\
(a) 75\quad(b) 150\quad(c) 210\quad(d) 243
\item[14.]Six cards and six envelopes are numbered 1,2,3,4,5,6 and card are to be placed in envelopes so that each envelope numbered 2. Then the number of ways it can be done is
\hfill{(JEE Adv.2014)}\\
(a) 264\quad(b) 265\quad(c) 53\quad(d) 67
\item[15.]A debate club consists of 6 girls and 4 boys.A team of 4 members is to be selected from this club including the selection of a captain(from among these 4 members) for the team. if the team has to include at most one boy,then the number of ways of selecting the taem is
\hfill{(JEE Adv.$2016$)}\\
(a) 380\quad(b) 320\quad(c) 260\quad(d) 95
\section*{D.MCQs with One or More than One Correct}
\item[1.]An n-digit number is a positive number with exactly n digits. Nine hundred distinct n-digit numbers are to be formed using only the three digits2,5 and 7. The smallest value of n for which  this is possible is
\hfill{($1998-2$ Marks)}\\
(a) 6\quad(b) 7\quad(c) 8\quad(d) 9\\
 \section*{E.Subjective Problems}   
\item[1.]Six X's have to be placed in the squares of figure below in such a way that each row contain at least one X. In how many different ways can this be done.
\hfill{($1978$)}\\
\begin{center}
\begin{tabular}{|c|c|}
\hline
\quad&\quad  \\
\end{tabular}
\end{center}
\begin{center}
\begin{tabular}{|c|c|c|c|}
\hline
    \quad&\quad&\quad &\quad\\
    \hline
\end{tabular}
\end{center}
\begin{center}
\begin{tabular}{|c|c|}
   \quad& \quad\\
   \hline
\end{tabular}\\
\end{center}
\end{enumerate}
\end{document}