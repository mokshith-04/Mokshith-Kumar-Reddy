%iffalse
\documentclass[journal]{IEEEtran}
\usepackage[a5paper, margin=10mm, onecolumn]{geometry}
%\usepackage{lmodern} % Ensure lmodern is loaded for pdflatex
\usepackage{tfrupee} % Include tfrupee package

\setlength{\headheight}{1cm} % Set the height of the header box
\setlength{\headsep}{0mm}     % Set the distance between the header box and the top of the text

\usepackage{gvv-book}
\usepackage{gvv}
\usepackage{cite}
\usepackage{amsmath,amssymb,amsfonts,amsthm}
\usepackage{algorithmic}
\usepackage{graphicx}
\usepackage{textcomp}
\usepackage{xcolor}
\usepackage{txfonts}
\usepackage{listings}
\usepackage{enumitem}
\usepackage{mathtools}
\usepackage{gensymb}
\usepackage{comment}
\usepackage[breaklinks=true]{hyperref}
\usepackage{tkz-euclide} 
\usepackage{listings}
% \usepackage{gvv}                                        
\def\inputGnumericTable{}                                 
\usepackage[latin1]{inputenc}                                
\usepackage{color}                                            
\usepackage{array}                                            
\usepackage{longtable}                                       
\usepackage{calc}                                             
\usepackage{multirow}                                         
\usepackage{hhline}                                           
\usepackage{ifthen}                                           
\usepackage{lscape}
\begin{document}
\bibliographystyle{IEEEtran}
\title{2021-August Session-31-08-2021 shift 1}
\author{EE24BTECH11009-Mokshith}
{\let\newpage\relax\maketitle}
\renewcommand{\thefigure}{\theenumi}
\renewcommand{\thetable}{\theenumi}
\setlength{\intextsep}{10pt} % Space between text and floats
\numberwithin{equation}{enumi}
\numberwithin{figure}{enumi}
\renewcommand{\thetable}{\theenumi}

\begin{enumerate}[start=16]
\item $96\cos \frac{\pi}{33} \cos\frac{2\pi}{33} \cos\frac{4\pi}{33} \cos\frac{8\pi}{33} \cos\frac{16\pi}{33}$
is equal to:
\begin{enumerate}
    \item $4$
    \item $2$
    \item $3$
    \item $1$
\end{enumerate}
\item Let the complex number $z = x + iy$ be such that 
$\frac{2z - 3i}{2z + i}$ is purely imaginary. If $x + y^2 = 0$, then $y^4 + y^2 - y$ is equal to:
\begin{enumerate}
    \item $\frac{3}{2}$
    \item $\frac{2}{3}$
    \item $\frac{4}{3}$
    \item $\frac{3}{4}$
\end{enumerate}
\item If $f(x) = \frac{\brak{\tan 1^\degree} x + \log_e\brak{123}}{x \log_e\brak{1234}-\brak{\tan 1^\degree}}$ and $x > 0$, then the least value of 
$f\brak{f\brak{x}} + f\brak{f\brak{\frac{4}{x}}}$ is:
\begin{enumerate}
    \item $2$
    \item $4$
    \item $8$
    \item $0$
\end{enumerate}
\item The slope of the tangent at any point $\brak{x, y}$ on a curve $y = y\brak{x}$ is 
$\frac{x^2 + y^2}{2xy}$, $x > 0$. If $y\brak{2} = 0$, then a value of $y\brak{8}$ is:
\begin{enumerate}
    \item $4\sqrt{3}$
    \item $-4\sqrt{2}$
    \item $-2\sqrt{3}$
    \item $2\sqrt{3}$
\end{enumerate}
\item Let the ellipse $E: x^2 + 9y^2 = 9$ intersect the positive $x$ and $y$-axes at the points $\vec{A}$ and $\vec{B}$ respectively. Let the major axis of $E$ be a diameter of the circle $C$. Let the line passing through $\vec{A}$ and $\vec{B}$ meet the circle $C$ at the point $\vec{P}$. If the area of the triangle with vertices $\vec{A}$, $\vec{P}$, and the origin $\vec{O}$ is $\frac{m}{n}$, where $m$ and $n$ are coprime, then $m - n$ is equal to:
\begin{enumerate}
    \item $16$
    \item $15$
    \item $18$
    \item $17$
\end{enumerate}

\section*{Section B}
\item Some couples participated in a mixed doubles badminton tournament. If the number of matches played, so that no couple plays in a match, is 840, then the total number of persons, who participated in the tournament, is \underline{\hspace{1cm}}.
\item The number of elements in the set $\cbrak{n \in \mathbb{Z} :\abs{ n^2 - 10n + 19} < 6 }$ is \underline{\hspace{1cm}}.
\item The number of permutations of the digits $1, 2, 3, \dots, 7$ without repetition, which neither contain the string $153$ nor the string $2467$, is \underline{\hspace{1cm}}.
\item Let $f: \brak{-2, 2} \to \mathbb{R}$ be defined by
$f(x) = \begin{cases}
x\sbrak{x} & , -2 < x < 0 \\
(x-1)\sbrak{x} & , 0 \leq x < 2
\end{cases}$
where $\sbrak{x}$ denotes the greatest integer function. If $m$ and $n$ respectively are the number of points in $\brak{-2, 2}$ at which $y =\abs{f\brak{x}}$ is not continuous and not differentiable, then $m + n$ is equal to \underline{\hspace{1cm}}.
\item Let a common tangent to the curves $y^2 = 4x$ and $\brak{x - 4}^2 + y^2 = 16$ touch the curves at the points $\vec{P}$ and $\vec{Q}$. Then $\brak{PQ}^2$ is equal to \underline{\hspace{1cm}}.
\item If the mean of the frequency distribution
\begin{table}[h!]
\centering
\begin{tabular}{|c|c|c|c|c|c|}
\hline
Class: & 0-10 & 10-20 & 20-30 & 30-40 & 40-50 \\
\hline
Frequency: & 2 & 3 & $x$ & 5 & 4 \\
\hline
\end{tabular}
\end{table}
is $28$, then its variance is \underline{\hspace{1cm}}.
\item The coefficient of $x^7$ in $\brak{1 - x + 2x^3}^{10}$ is \underline{\hspace{1cm}}.
\item Let $y = p\brak{x}$ be the parabola passing through the points $\brak{-1, 0}, \brak{0, 1}$ and $\brak{1, 0}$. If the area of the region 
$\cbrak{\brak{x, y} : \brak{x+1}^2+\brak{y-1}^2 \leq 1, y \leq p\brak{x}}$ is $A$, then $12\brak{\pi - 4A}$ is equal to \underline{\hspace{1cm}}.
\item Let $a, b, c$ be three distinct positive real numbers such that 
$\brak{2a}^{\log_e a} = \brak{bc}^{\log_e b}$ and $\brak{b}^{\log_e 2} = \brak{a}^{log_e c}$
Then $6a + 5bc$ is equal to \underline{\hspace{1cm}}.
\item The sum of all those terms, of the arithmetic progression $3, 8, 13, \dots, 373$, which are not divisible by $3$, is equal to \underline{\hspace{1cm}}.
\end{enumerate}
\end{document}
