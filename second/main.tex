\documentclass[journal]{IEEEtran}
\usepackage[a5paper, margin=10mm, onecolumn]{geometry}
%\usepackage{lmodern} % Ensure lmodern is loaded for pdflatex
\usepackage{tfrupee} % Include tfrupee package

\setlength{\headheight}{1cm} % Set the height of the header box
\setlength{\headsep}{0mm}     % Set the distance between the header box and the top of the text

\usepackage{gvv-book}
\usepackage{gvv}
\usepackage{cite}
\usepackage{amsmath,amssymb,amsfonts,amsthm}
\usepackage{algorithmic}
\usepackage{graphicx}
\usepackage{textcomp}
\usepackage{xcolor}
\usepackage{txfonts}
\usepackage{listings}
\usepackage{enumitem}
\usepackage{mathtools}
\usepackage{gensymb}
\usepackage{comment}
\usepackage[breaklinks=true]{hyperref}
\usepackage{tkz-euclide} 
\usepackage{listings}
% \usepackage{gvv}                                        
\def\inputGnumericTable{}                                 
\usepackage[latin1]{inputenc}                                
\usepackage{color}                                            
\usepackage{array}                                            
\usepackage{longtable}                                       
\usepackage{calc}                                             
\usepackage{multirow}                                         
\usepackage{hhline}                                           
\usepackage{ifthen}                                           
\usepackage{lscape}
\begin{document}
\bibliographystyle{IEEEtran}
\title{2009-CE-'25-36'}
\author{EE24BTECH11009-Mokshith}
{\let\newpage\relax\maketitle}
\renewcommand{\thefigure}{\theenumi}
\renewcommand{\thetable}{\theenumi}
\setlength{\intextsep}{10pt} % Space between text and floats
\numberwithin{equation}{enumi}
\numberwithin{figure}{enumi}
\renewcommand{\thetable}{\theenumi
}
\begin{enumerate}[start=25]
\item In the solution of the following set of linear equations by Gauss elimination using partial pivoting
    $5x + y + 2z = 34;$ $4y - 3z = 12;$  and  $10x - 2y + z = -4;$
    the pivots for elimination of $x$ and $y$ are
\begin{enumerate}
    \item 10 and 4
    \item 10 and 2
    \item 5 and 4
    \item 5 and -4
\end{enumerate}
\item The standard normal probability function can be approximated as
$$F(x_N) = \frac{1}{1 + \exp(-1.7255 x_N |x_N|^{0.12})}$$
where $x_N =$ standard normal deviate. If mean and standard deviation of annual precipitation are $102 cm$ and $27 cm$ respectively, the probability that the annual precipitation will be between $90 cm$ and $102 cm$ is
\begin{enumerate}
    \item 66.7 \%
    \item 50.0 \%
    \item 33.3 \%
    \item 16.7 \%
\end{enumerate}
\item Consider the following statements:
\begin{enumerate}[label=\Roman{*}.]
    \item On a principal plane, only normal stress acts.
    \item On a principal plane, both normal and shear stresses act.
    \item On a principal plane, only shear stress acts.
    \item Isotropic state of stress is independent of frame of reference.
\end{enumerate}
The TRUE statements are
\begin{enumerate}
    \item I and IV
    \item II
    \item II and IV
    \item II and III
\end{enumerate}
\item The degree of static indeterminacy of a rigidly jointed frame in a horizontal plane and subjected to vertical loads only, as shown in figure below, is
\begin{figure}[!ht]
\centering
\resizebox{3cm}{!}{
\begin{circuitikz}
\tikzstyle{every node}=[font=\normalsize]
\draw (5,13.5) to[short] (6.5,12);
\draw (6.5,12) to[short] (8,13.5);
\draw (8,13.5) to[short] (6.5,15);
\draw (5,13.5) to[short] (5,13);
\draw (5,13) to[short] (6.5,11.5);
\draw (6.5,12) to[short] (6.5,11.5);
\draw (6.5,11.5) to[short] (8,13);
\draw (8,13.5) to[short] (8,13);
\draw (6.5,15) to[short] (6,14.5);
\draw (5,13.5) to[short] (5.5,14);
\draw (5.5,14) to[short] (6.5,13);
\draw (6.5,13) to[short] (7,13.5);
\draw (6,14.5) to[short] (7,13.5);
\draw (6,14.5) to[short] (6,14);
\draw (0.5,11.25) to[short] (0.5,11.25);
\draw (6,14) to[short] (6.75,13.25);
\draw (3.75,14.25) to[short] (7,17.5);
\draw (7,17.5) to[short] (7,14.5);
\draw (3.75,10.25) to[short] (5.75,12.25);
\draw (6.25,13.25) to[short] (6.5,13.5);
\draw (3.75,10.25) to[short] (3.75,14.25);
\draw [->, >=Stealth] (7.25,13.5) -- (7.25,13);
\draw [->, >=Stealth] (4.5,16.5) -- (6.5,14.5);
\draw [->, >=Stealth] (4.5,16.5) -- (5.25,13.5);
\draw [->, >=Stealth] (6.25,13.5) -- (6.25,13);
\draw [->, >=Stealth] (7.25,14.5) -- (7.25,14);
\node [font=\normalsize] at (4,16.75) {End clamped to rigid wall};
\end{circuitikz}
}
\label{fig:my_label}
\end{figure}
\begin{enumerate}
    \item $6$
    \item $4$
    \item $3$
    \item $1$
\end{enumerate}
\item  A $12 mm$ thick plate is connected to two $8 mm$ thick plates, on either side through a $16 mm$ diameter power driven field rivet as shown in the figure below. Assuming permissible shear stress as $90 MPa$ and permissible bearing stress as $270 MPa$ in the rivet, the rivet value of the joint is
\begin{figure}[!ht]
\centering
\resizebox{4cm}{!}{%
\begin{circuitikz}
\tikzstyle{every node}=[font=\normalsize]
\draw (6,10.25) to[short] (9.25,10.25);
\draw (6,9.75) to[short] (9.25,9.75);
\draw (6,9) to[short] (9.25,9);
\draw (6,8.5) to[short] (9.25,8.5);
\draw (8.5,9.75) to[short] (8.5,9);
\draw (9.25,9.75) to[short] (12.5,9.75);
\draw (9.25,9) to[short] (12.5,9);
\draw (9.25,10.25) to[short] (9.25,9.75);
\draw (9.25,8.5) to[short] (9.25,9);
\draw (7.25,7.75) to[short] (7.25,7.5);
\draw (6,10.25) to[short] (6,9.75);
\draw (6,9) to[short] (6,8.5);
\draw (12.5,9.75) to[short] (12.5,9);
\draw [->, >=Stealth] (6,10) -- (5.5,10);
\draw [->, >=Stealth] (6,8.75) -- (5.5,8.75);
\draw [->, >=Stealth] (6.75,10.75) -- (6.75,10.25);
\draw [->, >=Stealth] (6.75,9.5) -- (6.75,9.75);
\draw [->, >=Stealth] (6.75,9.25) -- (6.75,9);
\draw [->, >=Stealth] (6.75,8) -- (6.75,8.5);
\draw [->, >=Stealth] (11.25,10.25) -- (11.25,9.75);
\draw [->, >=Stealth] (11.25,8.5) -- (11.25,9);
\draw [dashed] (9,10.25) -- (9,8.5);
\draw [dashed] (8.75,10.25) -- (8.75,8.5);

\node [font=\small] at (6.75,10) {$8 mm$};
\node [font=\small] at (6.75,8.75) {$8 mm$};
\node [font=\normalsize] at (11.5,9.5) {12 mm};
\node [font=\small] at (5.25,10) {$p/2$};
\node [font=\small] at (5.25,8.75) {$p/2$};
\draw [->, >=Stealth] (12.5,9.25) -- (13.5,9.25);
\node [font=\normalsize] at (13.75,9.25) {$p$};
\end{circuitikz}
}%

\label{fig:my_label}
\end{figure}
\begin{enumerate}
    \item $56.70 kN$
    \item $43.29 kN$
    \item $36.19 kN$
    \item $21.65 kN$
\end{enumerate}
\item  A hollow circular shaft has an outer diameter of $100 mm$ and a wall thickness of $25 mm$. The allowable shear stress in the shaft is $125 MPa$. The maximum torque the shaft can transmit is
\begin{enumerate}
    \item $46 kN m$
    \item $24.5 kN m$
    \item $23 kN m$
    \item $11.5 kN m$
\end{enumerate}
\item Consider the following statements for a compression member:
\begin{enumerate}
    \item The elastic critical stress in compression increases with decrease in slenderness ratio.
    \item The effective length depends on the boundary conditions at its ends.
    \item The elastic critical stress in compression is independent of the slenderness ratio.
    \item The ratio of the effective length to its radius of gyration is called as slenderness ratio.
\end{enumerate}

The TRUE statements are
\begin{enumerate}
    \item II and III
    \item III and IV
    \item II, III and IV
    \item I, II and IV
\end{enumerate}
\item Group I gives the shear force diagrams and Group II gives the diagrams of beams with supports an loading. Match the Group I with Group II.
\begin{figure}[!ht]
\centering
\resizebox{5cm}{!}{%
\begin{circuitikz}
\tikzstyle{every node}=[font=\huge]
\draw (3.5,8.5) to[short] (10.25,8.5);
\draw (3.5,8.5) to[short] (4.5,7.5);
\draw (4.5,7.5) to[short] (4.5,10.75);
\draw (4.5,10.75) to[short] (9.25,6);
\draw (9.25,6) to[short] (9.25,9.5);
\draw (9.25,9.5) to[short] (10.25,8.5);
\draw (3.5,4.25) to[short] (10.5,4.25);
\draw (3.5,4.25) to[short] (4.5,3.25);
\draw (4.5,3.25) to[short] (4.5,4.25);
\draw (10.5,4.25) to[short] (9.5,5.25);
\draw (9.5,5.25) to[short] (9.5,4.25);
\draw (3.75,0) to[short] (10.75,0);
\draw (3.75,0) to[short] (3.75,-1);
\draw (3.75,-1) to[short] (5,-1);
\draw (5,-1) to[short] (5,1);
\draw (5,1) to[short] (7.25,1);
\draw (7.25,1) to[short] (7.25,-1);
\draw (7.25,-1) to[short] (9.5,-1);
\draw (9.5,-1) to[short] (9.5,1);
\draw (9.5,1) to[short] (10.75,1);
\draw (10.75,1) to[short] (10.75,0);
\draw (3.75,-4.5) to[short] (11.25,-4.5);
\draw (3.75,-4.5) to[short] (3.75,-5.5);
\draw (3.75,-5.5) to[short] (5.25,-5.5);
\draw (5.25,-5.5) to[short] (5.25,-4.5);
\draw (10,-4.5) to[short] (10,-3.25);
\draw (10,-3.25) to[short] (11.25,-3.25);
\draw (11.25,-3.25) to[short] (11.25,-4.5);
\draw (2.25,-11.75) to[short] (14.75,-11.75);
\draw (2.25,-14.5) to[short] (14.75,-14.5);
\draw (2.25,-17.25) to[short] (14.75,-17.25);
\draw (2.25,-20.25) to[short] (15,-20.25);
\draw (4,-11.75) to[short] (3.5,-12.25);
\draw (4,-11.75) to[short] (4.5,-12.25);
\draw (3.25,-12.25) to[short] (4.75,-12.25);
\draw (3.25,-12.25) to[short] (3.25,-12.75);
\draw (3.25,-12.75) to[short] (4.75,-12.75);
\draw (4.75,-12.25) to[short] (4.75,-12.75);
\draw (4,-14.5) to[short] (3.5,-15);
\draw (4,-14.5) to[short] (4.5,-15);
\draw (3.25,-15) to[short] (4.75,-15);
\draw (3.25,-15) to[short] (3.25,-15.5);
\draw (3.25,-15.5) to[short] (4.75,-15.5);
\draw (4.75,-15) to[short] (4.75,-15.5);
\draw (4,-17.25) to[short] (3.5,-17.75);
\draw (4,-17.25) to[short] (4.5,-17.75);
\draw (3.25,-17.75) to[short] (4.75,-17.75);
\draw (3.25,-17.75) to[short] (3.25,-18.25);
\draw (3.25,-18.25) to[short] (4.75,-18.25);
\draw (4.75,-18.25) to[short] (4.75,-17.75);
\draw (4,-20.25) to[short] (3.5,-20.75);
\draw (4,-20.25) to[short] (4.5,-20.75);
\draw (3.25,-20.75) to[short] (4.75,-20.75);
\draw (3.25,-20.75) to[short] (3.25,-21.25);
\draw (3.25,-21.25) to[short] (4.75,-21.25);
\draw (4.75,-21.25) to[short] (4.75,-20.75);
\draw (13.25,-11.75) to[short] (12.75,-12.25);
\draw (13.25,-11.75) to[short] (13.75,-12.25);
\draw (12.5,-12.25) to[short] (14,-12.25);
\draw (12.5,-12.25) to[short] (12.5,-12.75);
\draw (12.5,-12.75) to[short] (14,-12.75);
\draw (14,-12.75) to[short] (14,-12.25);
\draw (13.5,-14.5) to[short] (13,-15);
\draw (13.5,-14.5) to[short] (14,-15);
\draw (12.75,-15) to[short] (14.25,-15);
\draw (12.75,-15) to[short] (12.75,-15.5);
\draw (12.75,-15.5) to[short] (14.25,-15.5);
\draw (14.25,-15.5) to[short] (14.25,-15);
\draw (13.75,-17.25) to[short] (13.25,-17.75);
\draw (13.75,-17.25) to[short] (14.25,-17.75);
\draw (13,-17.75) to[short] (14.5,-17.75);
\draw (13,-17.75) to[short] (13,-18.25);
\draw (13,-18.25) to[short] (14.5,-18.25);
\draw (14.5,-18.25) to[short] (14.5,-17.75);
\draw (13.75,-20.25) to[short] (13.25,-20.75);
\draw (13.75,-20.25) to[short] (14.25,-20.75);
\draw (13,-20.75) to[short] (14.25,-20.75);
\draw (13.5,-20.75) to[short] (14.5,-20.75);
\draw (13,-20.75) to[short] (13,-21.25);
\draw (13,-21.25) to[short] (14.5,-21.25);
\draw (14.5,-21.25) to[short] (14.5,-20.75);
\draw [<->, >=Stealth] (4,-13) -- (2.25,-13);
\draw [<->, >=Stealth] (4,-13) -- (13.25,-13);
\draw [<->, >=Stealth] (14.75,-13) -- (13.25,-13);
\draw [<->, >=Stealth] (2.25,-15.75) -- (4,-15.75);
\draw [<->, >=Stealth] (4,-15.75) -- (13.5,-15.75);
\draw [<->, >=Stealth] (13.5,-15.75) -- (14.75,-15.75);
\draw [<->, >=Stealth] (14.75,-18.5) -- (13.75,-18.5);
\draw [<->, >=Stealth] (13.75,-18.5) -- (4,-18.5);
\draw [<->, >=Stealth] (4,-18.5) -- (2.25,-18.5);
\draw [<->, >=Stealth] (4,-21.5) -- (2.25,-21.5);
\draw [<->, >=Stealth] (4,-21.5) -- (13.75,-21.5);
\draw [<->, >=Stealth] (13.75,-21.5) -- (15,-21.5);
\draw [<->, >=Stealth] (2.25,-11.25) -- (4,-11.25);
\draw [<->, >=Stealth] (13.25,-11.25) -- (14.75,-11.25);
\draw [<->, >=Stealth] (2.25,-17) -- (14.75,-17);
\draw [->, >=Stealth] (2.25,-19.5) .. controls (2.25,-20) and (2.25,-19.75) .. (2.25,-20.25) ;
\draw [->, >=Stealth] (15,-19.5) -- (15,-20.25);
\draw [->, >=Stealth] (2.25,-14) -- (2.25,-14.5);
\draw [->, >=Stealth] (14.75,-14) -- (14.75,-14.5);
\draw [->, >=Stealth] (8.75,-14) -- (8.75,-14.5);
\node [font=\Large] at (3.75,9.5) {$ql/2$};
\node [font=\huge] at (4.25,8.25) {-};
\node [font=\LARGE] at (5.25,9.25) {+};
\node [font=\Large] at (5.25,8) {$ql/4$};
\node [font=\huge] at (8.5,7.75) {-};
\node [font=\Large] at (9.5,9) {+};
\node [font=\Large] at (8.5,9.5) {$ql/4$};
\node [font=\Large] at (10,7) {$ql/2$};
\node [font=\LARGE] at (2.5,4) {Q.};
\node [font=\LARGE] at (2.5,0) {R.};
\node [font=\huge] at (4.5,-0.5) {-};
\node [font=\LARGE] at (6,0.5) {+};
\node [font=\huge] at (8.25,-0.5) {-};
\node [font=\LARGE] at (10,0.5) {+};
\node [font=\LARGE] at (9.75,4.75) {+};
\node [font=\huge] at (4.25,4) {-};
\node [font=\Large] at (5,3) {$ql/4$};
\node [font=\Large] at (8.5,5.25) {$ql/4$};
\node [font=\Large] at (4.5,0.5) {$q/2$};
\node [font=\Large] at (5.5,-0.5) {$q/2$};
\node [font=\Large] at (6.75,-1) {$q/2$};
\node [font=\Large] at (7.75,0.75) {$q/2$};
\node [font=\Large] at (10,-1) {$q/2$};
\node [font=\Large] at (11.25,0.75) {$q/2$};
\node [font=\LARGE] at (2.5,8.5) {P.};
\node [font=\Large] at (6,-5) {$q/2$};
\node [font=\Large] at (9,-3.5) {$q/2$};
\node [font=\LARGE] at (10.5,-3.75) {$+$};
\node [font=\Huge] at (4.5,-5) {$-$};
\node [font=\Large] at (4.25,-10.75) {$q/unitlength$};
\node [font=\Large] at (15,-10.75) {$q/unitlength$};
\node [font=\Large] at (3.25,-13.25) {$l/4$};
\node [font=\Large] at (8.5,-12.75) {$l$};
\node [font=\Large] at (14,-13.25) {$l/4$};
\node [font=\Large] at (9,-13.75) {$q$};
\node [font=\Large] at (1.75,-14) {$q/2$};
\node [font=\Large] at (15.5,-14) {$q/2$};
\node [font=\Large] at (3,-16) {$l/4$};
\node [font=\Large] at (8.5,-15.5) {$l$};
\node [font=\Large] at (14,-16) {$l/4$};
\node [font=\Large] at (8.25,-16.75) {$q/unitlength$};
\node [font=\Large] at (3,-18.75) {$l/4$};
\node [font=\Large] at (8.5,-18.25) {$l$};
\node [font=\Large] at (14.25,-18.75) {$l/4$};
\node [font=\LARGE] at (2.75,-19.75) {$q/2$};
\node [font=\LARGE] at (14.25,-19.75) {$q/2$};
\node [font=\Large] at (8.5,-21.25) {$l$};
\node [font=\Large] at (3,-21.75) {$l/4$};
\node [font=\Large] at (14.25,-21.75) {$l/4$};
\node [font=\Huge] at (6.5,12.25) {$Group I$};
\node [font=\Huge] at (7.5,-8.5) {$Group II$};
\node [font=\huge] at (2.5,-5) {$S.$};
\node [font=\huge] at (0.25,-11.75) {$1.$};
\node [font=\huge] at (0.25,-14.25) {$2.$};
\node [font=\huge] at (0.25,-17) {$3.$};
\node [font=\huge] at (0.25,-20) {$4.$};
\end{circuitikz}
}%

\label{fig:my_label}
\end{figure}
\begin{enumerate}
    \item P-3, Q-1, R-2, S-4
    \item P-3, Q-4, R-2, S-1
    \item P-2, Q-1, R-4, S-3
    \item P-2, Q-4, R-3, S-4
\end{enumerate}
\item A rectangular concrete beam of width $120 mm$ and depth $200 mm$ is prestressed by pretensioning to a force of $150 kN$ at an eccentricity of $20 mm$. The cross sectional area of the prestressing steel is $187.5 mm^2$. Take modulus of elasticity of steel and concrete as $2.1\times10^5 MPa$ and $3.0\times10^4 MPa$ respectively. The percentage loss of stress in the prestressing steel due to elastic deformation of concrete is
\begin{enumerate}
    \item 8.75
    \item 6.125
    \item 4.81
    \item 2.19
\end{enumerate}
\item Column I gives a list of test methods for evaluating properties of concrete and Column II gives the list of properties.
\begin{table}[H]
    \centering
    \begin{tabular}{|c|c|}
        \hline
        Point & Coordinates\\
        \hline
        $A$ & \myvec{0\\6}\\
        \hline
        $B$ & \myvec{8\\0}\\
        \hline
        $C$ & \myvec{0\\0}\\
        \hline
\end{tabular}

    \caption{}
    \label{tab:my_label}
\end{table}
The correct match of the test with the property is
\begin{enumerate}
    \item P-2, Q-4, R-1, S-3
    \item P-2, Q-1, R-4, S-3
    \item P-2, Q-4, R-3, S-1
    \item P-4, Q-3, R-1, S-2
\end{enumerate}
\item The laboratory test results of a soil sample are given below:
\begin{center}
\begin{tabular}{l l}
    Percentage finer than $4.75 mm$ & = $60$ \\
    Percentage finer than $0.075 mm $& = $30$ \\
    Liquid Limit & = $35$\% \\
    Plastic Limit & = $27$\%
\end{tabular}
\end{center}
The soil classification is
\begin{enumerate}
    \item $GM$
    \item $SM$
    \item $GC$
    \item $ML-MI$
\end{enumerate}
\item A plate load test is carried out on a $300 mm \times 300 mm$ plate placed at $2 m$ below the ground level to determine the bearing capacity of a $2 m\times 2 m$ footing placed at same depth of $2 m$ on a homogeneous sand deposit extending $10 m$ below ground. The ground water table is $3 m$ below the ground level. Which of the following factors \underline{does not} require a correction to the bearing capacity determined based on the load test?
\begin{enumerate}
    \item Absence of the overburden pressure during the test
    \item Size of the plate is much smaller than the footing size
    \item Influence of the ground water table
    \item Settlement is recorded only over a limited period of one or two days
\end{enumerate}
\end{enumerate}
\end{document}
