%iffalse
\documentclass[journal]{IEEEtran}
\usepackage[a5paper, margin=10mm, onecolumn]{geometry}
%\usepackage{lmodern} % Ensure lmodern is loaded for pdflatex
\usepackage{tfrupee} % Include tfrupee package

\setlength{\headheight}{1cm} % Set the height of the header box
\setlength{\headsep}{0mm}     % Set the distance between the header box and the top of the text

\usepackage{gvv-book}
\usepackage{gvv}
\usepackage{cite}
\usepackage{amsmath,amssymb,amsfonts,amsthm}
\usepackage{algorithmic}
\usepackage{graphicx}
\usepackage{textcomp}
\usepackage{xcolor}
\usepackage{txfonts}
\usepackage{listings}
\usepackage{enumitem}
\usepackage{mathtools}
\usepackage{gensymb}
\usepackage{comment}
\usepackage[breaklinks=true]{hyperref}
\usepackage{tkz-euclide} 
\usepackage{listings}
% \usepackage{gvv}                                        
\def\inputGnumericTable{}                                 
\usepackage[latin1]{inputenc}                                
\usepackage{color}                                            
\usepackage{array}                                            
\usepackage{longtable}                                       
\usepackage{calc}                                             
\usepackage{multirow}                                         
\usepackage{hhline}                                           
\usepackage{ifthen}                                           
\usepackage{lscape}
\begin{document}
\bibliographystyle{IEEEtran}
\title{2024-April Session-06-04-2024 shift 2}
\author{EE24BTECH11009-Mokshith}
{\let\newpage\relax\maketitle}
\renewcommand{\thefigure}{\theenumi}
\renewcommand{\thetable}{\theenumi}
\setlength{\intextsep}{10pt} % Space between text and floats
\numberwithin{equation}{enumi}
\numberwithin{figure}{enumi}
\renewcommand{\thetable}{\theenumi}
\begin{enumerate}[start=16]
\item If three letters can be posted to any one of the $5 $ different addresses, then the probability that the three are posted to exactly two addresses is:
    \begin{enumerate}
    \item $\frac{12}{25}$
    \item $\frac{4}{25}$
    \item $\frac{6}{25}$
    \item $\frac{18}{25}$
\end{enumerate}
\item If the locus of the point, whose distances from the point $\brak{2, 1}$ and $\brak{1, 3}$ are in the ratio $5: 4$, is
$ax^2+by^2+cxy+dx+ey+170=0$, then the value of $a^2+2b+3c+4d+e$ is equal to:
\begin{enumerate}
\item 5
\item -27
\item 437
\item 37
\end{enumerate}
\item A software company sets up $m$ number of computer systems to finish an assignment in $17$ days. If
$4$ computer systems crashed on the start of the second day, $4$ more computer systems crashed on
the start of the third day and so on, then it took $8$ more days to finish the assignment. The value of
$m$ is equal to:
\begin{enumerate}
\item 180
\item 125
\item 150
\item 160
\end{enumerate}
\item If $\int\frac{1}{a^{2}\sin^{2}x+b^{2}\cos^{2}x}dx=\frac{1}{12}\tan^{-1}(3\tan x)+$ constant, then the maximum value of $a\sin x+b\cos x$, is:

\begin{enumerate}
\item $\sqrt{42}$
\item $\sqrt{39}$
\item $\sqrt{41}$
\item $\sqrt{40}$
\end{enumerate}

\item Let $0\leq r\leq n$. If $\comb{n+1}{r+1}:\comb{n}{r}:\comb{n-1}{r-1}=55:35:21$, then $2n+5r$ is equal to:
\begin{enumerate}
\item 62
\item 60
\item 55
\item 50
\end{enumerate}
\item If the shortest distance between the lines $\frac{x-\lambda}{3}=\frac{y-2}{-1}=\frac{z-1}{1}$ and $\frac{x+2}{2\-3}=\frac{y+5}{2}=\frac{z-4}{4}$ is $\frac{44}{\sqrt{30}}$, then the largest possible value of $\abs{\lambda}$ is equal to:
\item Let $\sbrak{t}$ denote the largest integer less than or equal to $t$. If $\int_{0}^{3}\brak{\sbrak{x^{2}}+\sbrak{\frac{x^{2}}{2}}}dx=a+b\sqrt{2}-\sqrt{3}-\sqrt{5}+c\sqrt{6}-\sqrt{7}$, where $a, b, c\in\mathbb{Z}$, then $a+b+c$ is equal to:

\item Let $\alpha, \beta$ be roots of $x^{2}+\sqrt{2}x-8=0$. If $U_{n}=\alpha^{n}+\beta^{n}$, then $\frac{U_{10}+\sqrt{2}U_{9}}{2U_{8}}$ is equal to:

\item In a triangle $ABC$, $BC=7$, $AC=8$, $AB=\alpha\in\mathbb{N}$ and $\cos A=\frac{2}{3}$. If $49\cos\brak{3C}+42=\frac{m}{n}$, where $\gcd(m,n)=1$, then $m+n$ is equal to:

\item The length of the latus rectum and directrices of a hyperbola with eccentricity $e$ are $9$ and $x=\pm\frac{4}{\sqrt{3}}$, respectively. Let the line $y-\sqrt{3}x+\sqrt{3}=0$ touch this hyperbola at $\brak{x_{0},y_{0}}$. If $m$ is the product of the focal distances of the point $\brak{x_{0},y_{0}}$, then $4e^{2}+m$ is equal to:
\item If $S\brak{x}=\brak{1+x}+2\brak{1+x}^{2}+3\brak{1+x}^{3}+\cdots+60\brak{1+x}^{60}$ and $\brak{60}^2 S\brak{60}=a\brak{b}^{b}+b$, where $a, b\in\mathbb{N}$, then $a+b$ is equal to:
\item If the system of equations
\[2x+7y+\lambda z=3\]
\[3x+2y+5z=4\]
\[x+\mu y+32z=-1\]
has infinitely many solutions, then $\brak{\lambda-\mu}$ is equal to:
\item Let $\sbrak{t}$ denote the greatest integer less than or equal to $t$. Let $f:[0,\infty)\rightarrow \mathbb{R}$ be a function defined by
$f(x)=\sbrak{\frac{x}{2}+3}-\sbrak{\sqrt{x}}$
Let $S$ be the set of all points in the interval $\sbrak{0, 8}$ at which $f$ is not continuous. Then $\displaystyle \sum_{a\in S}a$ is equal to:
\item If the solution $y(x)$ of the given differential equation $\brak{e^{y}+1}\cos x dx+e^{y}\sin x dy=0$ passes through the point $\brak{\frac{\pi}{2},0}$, then the value of $e^{y\brak{\frac{\pi}{6}}}$ is equal to:
\item From a lot of $12$ items containing $3$ defectives, a sample of $5$ items is drawn at random. Let the random variable $X$ denote the number of defective items in the sample. Let items in the sample be drawn one by one without replacement. If variance of $X$ is $\frac{m}{n}$, where $\gcd(m,n)=1$, then $n-m$ is equal to:
\end{enumerate}
\end{document}