\let\negmedspace\undefined
\let\negthickspace\undefined
\documentclass[journal,12pt,twocolumn]{IEEEtran}
\usepackage{cite}
\usepackage{amsmath,amssymb,amsfonts,amsthm}
\usepackage{algorithmic}
\usepackage{graphicx}
\usepackage{textcomp}
\usepackage{xcolor}
\usepackage{txfonts}
\usepackage{listings}
\usepackage{enumitem}
\usepackage{mathtools}
\usepackage{gensymb}
\usepackage{comment}
\usepackage[breaklinks=true]{hyperref}
\usepackage{tkz-euclide} 
\usepackage{listings}
\usepackage{gvv}                                        
\def\inputGnumericTable{}                                 
\usepackage[latin1]{inputenc}                                
\usepackage{color}                                            
\usepackage{array}                                            
\usepackage{longtable}                                       
\usepackage{calc}                                             
\usepackage{multirow}                                         
\usepackage{hhline}                                           
\usepackage{ifthen}                                           
\usepackage{lscape}
\usepackage{multicol}

\newtheorem{theorem}{Theorem}[section]
\newtheorem{problem}{Problem}
\newtheorem{proposition}{Proposition}[section]
\newtheorem{lemma}{Lemma}[section]
\newtheorem{corollary}[theorem]{Corollary}
\newtheorem{example}{Example}[section]
\newtheorem{definition}[problem]{Definition}
\newcommand{\BEQA}{\begin{eqnarray}}
\newcommand{\EEQA}{\end{eqnarray}}
\newcommand{\define}{\stackrel{\triangle}{=}}
\theoremstyle{remark}
\newtheorem{rem}{Remark}
\begin{document}


\bibliographystyle{IEEEtran}
\vspace{3cm}

\title{4.Permutations and Combinations (sec:c,d,e)}
\author{EE24BTECH11009 - Mokshith Kumar Reddy$^{*}$% <-this % stops a space
}
\maketitle
\newpage
\bigskip
\begin{enumerate}[start=3]
\item  The value of the expression $\comb{47}{4}+\sum\limits_{j=1}^{5} \comb{52-j}{3}$ is equal to\hfill{(1982-2 Marks)}\\
\begin{multicols}{2} 
\begin{enumerate}
\item $^{47}C_5$
\item$^{52}C_4$
\columnbreak
\item $^{52}C_5$
\item none of these\\
\end{enumerate}
\end{multicols}
\item Eight chairs are numbered $1$ to $8$. Two women and three men wish to occupy one chair each. First the women choose the chairs from amongst the chair marked $1$to$4$; and then the men select the chairs from amongst the remaining. the number of possible arrangements is
\hfill{(1982-2 Marks)}\\
\begin{multicols}{2} 
\begin{enumerate}
\item $^6C_3\times^4C_2$
\item $^4C_2+^4P_3$\columnbreak
\item $^4P_2\times^4P_3$
\item none of these\\[4pt]
\end{enumerate}
\end{multicols}
\item A five-digit numbers divisible by $3$ is to be formed using the numeral $0,1,2,3,4$ and$ 5$, without repetition. the total number of ways this can be done is\\
\hspace*{\fill}(1989-2 Marks).
\begin{multicols}{2} 
\begin{enumerate}
\item $216$\item  600\columnbreak\item $240$\item 3125\\
\end{enumerate}
\end{multicols}
\item How many different nine digit numbers can be formed from the number $223355888$ by rearranging its digits so that the odd digits occupy even positions ?\\
\hspace*{\fill}(2000S)\\
\begin{multicols}{2} 
\begin{enumerate}
\item16 \item 60\columnbreak\item 36\item 4
\end{enumerate}
\end{multicols}
\item Let $T_n$ denote the number of triangles which can be formed using the vertices of a regular polygon of $n$ sides. If $T_{n+1}-T_n=21$, then n equals
\hfill{(2001S)}\\
\begin{multicols}{2} 
\begin{enumerate}
\item 5\item 6\columnbreak\item  7\item  4
\end{enumerate}
\end{multicols}
\item The number of arrangements of the letters of the word $BANANA$ in which the two $N$'s do not appear adjacently is\\
\hspace*{\fill}(2002S)
\begin{multicols}{2} 
\begin{enumerate}
\item  40\item 80\columnbreak\item  60\item  100
\end{enumerate}
\end{multicols}
\item A rectangle with sides of length $\brak{2m-1}$ and $\brak{2n-1}$ units is divided into squares of unit length by drawing parallel lines as shown in the diagram, then the number of rectangles possible with odd side lengths is\\
\hspace*{\fill}(2005S)
\begin{center}
\begin{tabular}{|c|c|c|c|}
\hline
\quad&\quad&\quad&\quad\\
\hline
\quad&\quad&\quad&\quad\\
\hline
\quad&\quad&\quad&\quad\\
\hline\quad&\quad&\quad&\quad\\
\hline\quad&\quad&\quad&\quad\\
\hline\quad&\quad&\quad&\quad\\
\hline
\end{tabular}
\end{center}
\begin{multicols}{2} 
\begin{enumerate}
\item  $(m+n-1)^2$
\item  $m^2n^2$\columnbreak\item  $4^{m+n-1}$\item $m(m+1)n(n+1)$
\end{enumerate}
\end{multicols}
\item If the LCM of$p,q$ is $r^2t^4s^2$ where $r,s,t$ are prime numbers and $p,q$ are the positive integers then the number of ordered pair$\brak{p,q}$ is \hfill{($2006$)}\\
\begin{multicols}{2} 
\begin{enumerate}
\item  252\item  225\columnbreak\item  254\item  224
\end{enumerate}
\end{multicols}
\item The letters of the word COCHIN are permuted and all the permutations are arranged in an alphabetical order as in an English dictionary. The number of words that appear before the word COCHIN is
\hfill{($2007-3 marks$)}\\
\begin{multicols}{2} 
\begin{enumerate}
\item  360\item  96\columnbreak\item 192\item  48
\end{enumerate}
\end{multicols}
\item The number of seven digit integers, with sum of the digits equal to 10 and formed by using the digits $1,2$,and$3$ only, is
\hfill{($2009$)}\\
\begin{multicols}{2} 
\begin{enumerate}
\item 55\item 77\columnbreak\item  66\item 88
\end{enumerate}
\end{multicols}
\item The total number of ways in which $5$ balls of different colours can be distributed among $3$ persons so that each person gets at least one ball is
\hfill{(2012)}\\
\begin{multicols}{2} 
\begin{enumerate}
\item  75\item 210\columnbreak\item150\item (d) 243
\end{enumerate}
\end{multicols}
\item cards and six envelopes are numbered $1,2,3,4,5,6$ and card are to be placed in envelopes so that each envelope numbered$2$. Then the number of ways it can be done is
\hspace*{\fill}(JEE Adv.2014)\\
\begin{multicols}{2} 
\begin{enumerate}
\item  264\item53 \columnbreak \item 265\item  67
\end{enumerate}
\end{multicols}
\item A debate club consists of $6 $girls and $4$ boys.A team of $4$ members is to be selected from this club including the selection of a captain (from among these 4 members) for the team. if the team has to include at most one boy,then the number of ways of selecting the team is
\hfill{(JEE Adv.$2016$)}\\
\begin{multicols}{2} 
\begin{enumerate}
\item  380\item 260\columnbreak\item 320\item  95
\end{enumerate}
\end{multicols}
\section*{D . MCQs with One or More than One Correct}
\item[1.] An n-digit number is a positive number with exactly n digits. Nine hundred distinct n-digit numbers are to be formed using only the three digits$2,5$ and $7$. The smallest value of n for which  this is possible is
\hfill{($1998-2$ Marks)}\\
\begin{multicols}{2} 
\begin{enumerate}
\item 6\item  8\columnbreak\item 7\item 9\\
\end{enumerate}
\end{multicols}
 \section*{E. Subjective Problems}   
\item[1.]Six $X$'s have to be placed in the squares of figure below in such a way that each row contain at least one $X$. In how many different ways can this be done.
\hfill{(s1978)}\\
\begin{center}
\begin{tabular}{|c|c|}
\hline
\quad&\quad  \\
\end{tabular}
\end{center}
\begin{center}
\begin{tabular}{|c|c|c|c|}
\hline
    \quad&\quad&\quad &\quad\\
    \hline
\end{tabular}
\end{center}
\begin{center}
\begin{tabular}{|c|c|}
   \quad& \quad\\
   \hline
\end{tabular}\\
\end{center}
\end{enumerate}
\end{document}