\let\negmedspace\undefined
\let\negthickspace\undefined
\documentclass[journal,12pt,onecolumn]{IEEEtran}
\usepackage{cite}
\usepackage{amsmath,amssymb,amsfonts,amsthm}
\usepackage{algorithmic}
\usepackage{graphicx}
\usepackage{textcomp}
\usepackage{xcolor}
\usepackage{txfonts}
\usepackage{listings}
\usepackage{enumitem}
\usepackage{mathtools}
\usepackage{gensymb}
\usepackage{comment}
\usepackage[breaklinks=true]{hyperref}
\usepackage{tkz-euclide} 
\usepackage{listings}
\usepackage{gvv}                                        
\def\inputGnumericTable{}                                 
\usepackage[latin1]{inputenc}                                
\usepackage{color}                                            
\usepackage{array}                                            
\usepackage{longtable}                                       
\usepackage{calc}                                             
\usepackage{multirow}                                         
\usepackage{hhline}                                           
\usepackage{ifthen}                                           
\usepackage{lscape}
\usepackage{multicol}

\newtheorem{theorem}{Theorem}[section]
\newtheorem{problem}{Problem}
\newtheorem{proposition}{Proposition}[section]
\newtheorem{lemma}{Lemma}[section]
\newtheorem{corollary}[theorem]{Corollary}
\newtheorem{example}{Example}[section]
\newtheorem{definition}[problem]{Definition}
\newcommand{\BEQA}{\begin{eqnarray}}
\newcommand{\EEQA}{\end{eqnarray}}
\newcommand{\define}{\stackrel{\triangle}{=}}
\theoremstyle{remark}
\newtheorem{rem}{Remark}
\begin{document}


\bibliographystyle{IEEEtran}
\vspace{3cm}

\title{JEE Main 2021 Paper - 31st September 2021 | Shift 1 (Maths)}
\author{EE24BTECH11009 - Mokshith Kumar Reddy$^{*}$% <-this % stops a space
}
\maketitle
\bigskip

\begin{enumerate}[start=16]
\item If $\frac{dy}{dx} = \frac{2^{x+y}-2^x}{2^y}$ with $y\brak{0}=1$, then $y\brak{1}$ is equal to:
\begin{enumerate}
\item $\log_2\brak{2 + e}$
\item $\log_2\brak{1 + e}$
\item $\log_2\brak{2e}$
\item $\log_2\brak{1 + e^2}$
\end{enumerate}
\item $\displaystyle \lim_{x \to 0} \frac{\sin^2\brak{\pi\cos 4x}}{x^4}$
is equal to:
\begin{enumerate}
\item $\pi^2$  
\item $2\pi^2$
\item $4\pi^2$
\item $4\pi$
\end{enumerate}
\item A vertical pole is divided in the ratio $3:7$ by a mark on it with lower part shorter than the upper part. If the lower part subtend equal angles at a point on the ground 18 m away from the base of the pole, then the height of the pole $\brak{\text{in meters}}$?
\begin{enumerate}
\item $12\sqrt{15}$
\item$12\sqrt{10}$
\item $8\sqrt{10}$
\item $6\sqrt{10}$
\end{enumerate}
\item If $a_r = \cos\left(\frac{2r\pi}{9}\right) + i\sin\left(\frac{2r\pi}{9}\right), r= 1,2,3,\cdots,i$ then the determinant 
$\mydet{
a_1 & a_2 & a_3 \\
a_4 & a_5 & a_6 \\
a_7 & a_8 & a_9}
$
 is equal to:
\begin{enumerate}
\item $a_2a_6 - a_4a_8$
\item $a_9$
\item $a_1a_9 - a_3a_7$
\item $a_5$
\end{enumerate}
\item The line $12x\cos\theta+5y\sin\theta = 60$ is tangent to which of the following curves?
\begin{enumerate}
\item $x^2 + y^2 = 169$
\item $144x^2 + 25y^2 = 3600$
\item $25x^2 + 12y^2 = 3600$
\item $x^2 + y^2 = 60$
\end{enumerate}
\end{enumerate}
\begin{enumerate}[start=1]
\section*{Section B}

\item Let $\sbrak{t}$ denote the greatest integer $\leq t$. Then the value of $8 \cdot \int_{\frac{-1}{2}}^{1} \left(\sbrak{2x} + \abs{x} \right)\, dx $
is.
\item A point $z$ moves in the complex plane such that 
$\arg\left(\frac{z - 2}{z + 2}\right) = \frac{\pi}{4}$, then the minimum value of $\abs{z - 9\sqrt{2} - 2i}^2$ is equal to.
\item  The square of the distance of the point of intersection of the line 
$\frac{x - 1}{2} = \frac{y - 2}{3} = \frac{z - 1}{6}$
and the plane $2x - y + z = 6$ from the point $\brak{-1, -1, 2}$ is.
\item If $R$ is the least value of $a$ such that the function $f(x) = x^2 + ax + 1$ is increasing on $\sbrak{1, 2}$ and $'S'$ is the greatest value of $'a'$ such that the function $f(x) = x^2 + ax + 1$ is decreasing on $\sbrak{1, 2}$, then the value of $\abs{R - S}$ is.
\item The mean of 10 numbers $7 \times 8, 10 \times 10, 13 \times 12, 16 \times 14, \dots$ is.
\item  If the variable line $3x + 4y = \alpha$ lies between the two circles $\brak{x - 1}^2 + \brak{y - 1}^2 = 1$ and $\brak{x - 9}^2 + \brak{y - 1}^2 = 4$, without intercepting a chord on either circle, then the sum of all the integral values of $\alpha$ is.
\item The number of six letter words (with or without meaning), formed using all the letters of the word 'VOWELS', such that all the consonants never come together, is.
\item If $x\phi\brak{x} =\int_{5}^{x}\left(3t^2 - 2\phi^\prime\brak{t}\right)\, dt, x > -2$
and $\phi\brak{0} = 4$, then $\phi\brak{2}$ is.
\item If $\left(\frac{3^6}{3^4}\right)k \text{ is the term independent of } x$, in the binomial expansion of $\left(\frac{x}{4} - \frac{12}{x^2}\right)^{12}
$ then $k$ is.
\item An electric instrument consists of two units. Each unit must function independently for the instrument to operate. The probability that the first unit functions is $0.9$ and that of the second unit is $0.8$. The instrument is switched on and it fails to operate. If the probability that only the first unit failed and the second unit is functioning is $p$, then $98p$ is equal to.
\end{enumerate}
\end{document}