\let\negmedspace\undefined
\let\negthickspace\undefined
\documentclass[journal,12pt,twocolumn]{IEEEtran}
\usepackage{cite}
\usepackage{amsmath,amssymb,amsfonts,amsthm}
\usepackage{algorithmic}
\usepackage{graphicx}
\usepackage{textcomp}
\usepackage{xcolor}
\usepackage{txfonts}
\usepackage{listings}
\usepackage{enumitem}
\usepackage{mathtools}
\usepackage{gensymb}
\usepackage{comment}
\usepackage[breaklinks=true]{hyperref}
\usepackage{tkz-euclide} 
\usepackage{listings}
\usepackage{gvv}                                        
\def\inputGnumericTable{}                                 
\usepackage[latin1]{inputenc}                                
\usepackage{color}                                            
\usepackage{array}                                            
\usepackage{longtable}                                       
\usepackage{calc}                                             
\usepackage{multirow}                                         
\usepackage{hhline}                                           
\usepackage{ifthen}                                           
\usepackage{lscape}
\usepackage{multicol}

\newtheorem{theorem}{Theorem}[section]
\newtheorem{problem}{Problem}
\newtheorem{proposition}{Proposition}[section]
\newtheorem{lemma}{Lemma}[section]
\newtheorem{corollary}[theorem]{Corollary}
\newtheorem{example}{Example}[section]
\newtheorem{definition}[problem]{Definition}
\newcommand{\BEQA}{\begin{eqnarray}}
\newcommand{\EEQA}{\end{eqnarray}}
\newcommand{\define}{\stackrel{\triangle}{=}}
\theoremstyle{remark}
\newtheorem{rem}{Remark}
\begin{document}


\bibliographystyle{IEEEtran}
\vspace{3cm}

\title{JEE Main 2020 Paper - 4th September 2020 | Shift 1 (Maths)}
\author{EE24BTECH11009 - Mokshith Kumar Reddy$^{*}$% <-this % stops a space
}
\maketitle
\newpage
\bigskip

\author{}
\date{}
\maketitle

\section*{Questions 1-15}

\begin{enumerate}

\item Let \(y=y(x)\) be the solution of the differential equation, \(xy^\prime-y=x^2(x\cos x+\sin x),x>0\). If \(y(\pi)=\pi\), then \(y'^\prime(\frac{\pi}{2})+y(\frac{\pi}{2})\) is equal to
    \begin{multicols}{2}
    \begin{enumerate}
        \item \( 2+\frac{\pi}{2}+\frac{\pi^2}{4} \)
        \item \( 2+\frac{\pi}{2} \)
        \item \( 1+\frac{\pi}{2} \)
        \item \( 
        1+
        \frac{\pi}{2}+\frac{\pi^2}{4} \)
    \end{enumerate}
    \end{multicols}
    
    \item The value of \( \sum_{r=0}^{20} \comb{50-r}{6}\) is equal to:
    \begin{multicols}{2}
    \begin{enumerate}
        \item \( \comb{51}{7}-\comb{30}{7} \)
        \item \( \comb{51}{7}+\comb{30}{7} \)
        \item \( \comb{50}{7}-\comb{30}{7} \)
        \item \( \comb{50}{6}-\comb{30}{6} \)
    \end{enumerate}
    \end{multicols}

    \item Let \(\lfloor t\rfloor\) denote the greatest integer \(\leq t\). Then the equation in $x$,$[x]^2+2[x+2]-7=0$ has:
    \begin{multicols}{2}
    \begin{enumerate}
        \item exactly four integral solutions
        \item infinitely many solutions
        \item no integral solution
        \item exactly two solutions
    \end{enumerate}
    \end{multicols}

    \item Let $P\brak{3,3}$ be a point on the hyperbola, $\frac{x^2}{a^2}-\frac{y^2}{b^2}=1$. If the normal to it at $P$ intersects the x-axis at $\brak{9,0}$ and $e$ is its eccentricity, then the ordered pair $\brak{a^2, e^2}$ is equal to:
\begin{multicols}{2}
\begin{enumerate}
    \item $\brak{9,3}$
    \item $\left(\frac{9}{2},2\right)$
    \item $\left(\frac{9}{2},3\right)$
    \item $\left(\frac{3}{2},2\right)$
\end{enumerate}
\end{multicols}
\item Let $\frac{x^2}{a^2}+\frac{y^2}{b^2}=1$ $\brak{a>b}$ be a given ellipse, the length of whose latus rectum is 10. If its eccentricity is the maximum value of the function, $\phi(t)=\frac{5}{12}+t-t^2$, then $a^2 + b^2$ is equal to
\begin{multicols}{2}
\begin{enumerate}
    \item 135
    \item 116
    \item 126
    \item 145
\end{enumerate}
\end{multicols}
\item Let $f(x)=\int\frac{\sqrt{x}}{1+x^2}\,dx$. Then $f(3)-f(1)$ is equal to:
\begin{multicols}{2}
\begin{enumerate}
    \item $-\frac{\pi}{6}+\frac{1}{2}+\frac{\sqrt{3}}{4}$
    \item $\frac{\pi}{6}+\frac{1}{2}-\frac{\sqrt{3}}{4}$
    \item $-\frac{\pi}{12}+\frac{1}{2}+\frac{\sqrt{3}}{4}$
    \item $\frac{\pi}{12}+\frac{1}{2}-\frac{\sqrt{3}}{4}$
\end{enumerate}
\end{multicols}
\item If $1+\brak{1-2^2\cdot1}+\brak{1-4^2\cdot3}+\brak{1-6^2\cdot5}+\cdots+\brak{1-20^2\cdot19}=\alpha-220\beta$, then an ordered pair $\brak{\alpha,\beta}$ is equal to:
\begin{multicols}{2}
\begin{enumerate}
    \item $\brak{10, 97}$
    \item $\brak{11, 103}$
    \item $\brak{11, 97}$
    \item $\brak{10,103}$
\end{enumerate}
\end{multicols}
\item The integral $\int\left(\frac{x}{x\sin x+\cos x}\right)^2\,dx$ is equal to (where $C$ is a constant of integration):
\begin{multicols}{2}
\begin{enumerate}
    \item $\tan x-\frac{x\sec x}{x\sin x+\cos x}+C$
    \item $\sec x+\frac{x\tan x}{x\sin x+\cos x}+C$
    \item $\sec x-\frac{x\tan x}{x\sin x+\cos x}+C$
    \item $\tan x+\frac{x\sec x}{x \sin x + \cos x}+C$
\end{enumerate}
\end{multicols}
\item Let $f(x)=|x-2|$ and $g(x)=f(f(x))$,$x\in [0, 4]$. Then $\int_0^3 \brak{g(x)-f(x)}\, dx$ is equal to:
\begin{multicols}{2}
\begin{enumerate}
    \item $\frac{1}{2}$
    \item 0
    \item 1
    \item $\frac{3}{2}$
\end{enumerate}
\end{multicols}
\item Let $x_0$ be the point of local maxima of $f(x)=\vec{a}\cdot\brak{\vec{b}\times\vec{c}}$, where $\vec{a}=x\hat{i}+2\hat{j}+3\hat{k}$,$\Vec{b}=-2\hat{i}+x\hat{j}-\hat{k}$, and $\vec{c}=7\hat{i}-2\hat{j}+x\hat{k}$. Then the value of $ \vec{a}\cdot\vec{b}+\vec{b}\cdot\vec{c}+\vec{c}\cdot\vec{a}$ at $x=x_0$ is:
\begin{multicols}{2}
\begin{enumerate}
    \item -22
    \item -4
    \item -30
    \item 14
\end{enumerate}
\end{multicols}
\item A triangle $ABC$ lying in the first quadrant has two vertices as $ A(1,2)$ and $B(3,1)$. If $\angle BAC=90^\circ$, and ar$\brak{\triangle ABC}$ is $5\sqrt{5}$, then the abscissa of the vertex $C$ is:
\begin{multicols}{2}
\begin{enumerate}
    \item $1+\sqrt{5}$
    \item $1+2\sqrt{5}$
    \item $2\sqrt{5}-1$
    \item $2+\sqrt{5}$
\end{enumerate}
\end{multicols}
\item Let f be a twice differentiable function on $\brak{1,6}$. If $f(2)=8, f^\prime{}{}(2)=5, f^\prime(x)\geq1$ and $f'^\prime(x)\geq4$, for all $x\in [1,6]$, then:
\begin{multicols}{2}
\begin{enumerate}
    \item $f(5)+f^\prime(5)\geq28$
    \item $f^\prime(5)+f'^\prime(5)\leq20$
    \item $f(5)\leq10$
    \item $f(5)+f^\prime(5)\leq26$
\end{enumerate}
\end{multicols}
\item Let $\alpha$ and $\beta$ be the roots of $x^2-3x+p=0$ and $\gamma$ and $\delta$ be the roots of $x^2-6x+q=0$. If $\alpha,\beta,\gamma,\delta$ form a geometric progression. Then ratio $\brak{2q+p}:\brak{2q-p}$ is:
\begin{multicols}{2}
\begin{enumerate}
    \item 33 :31
    \item 9 :7
    \item 3 :1
    \item 5 :3
\end{enumerate}
\end{multicols}
\item Let $u=\frac{2z+i}{z-ki},z=x+iy$ and $k>0$. If the curve represented by $Re(u)+Im(u)=1$ intersects the y-axis at the points P and Q where PQ=5, then the value of k is:
\begin{multicols}{2}
\begin{enumerate}
    \item 4
    \item $\frac{1}{2}$
    \item 2
    \item $\frac{3}{2}$
\end{enumerate}
\end{multicols}
\item If $A=\myvec{\cos\theta&i\sin\theta\\i\sin\theta&\cos\theta},\left(\theta=\frac{\pi}{24}\right)$ and $A^5=\myvec{a&b\\c&d}$,where $i=\sqrt{-1}$,then which one of the following is not true?
\begin{multicols}{2}
\begin{enumerate}
    \item $a^2-d^2=0$
    \item $a^2-c^2=1$
    \item $0\leq a^2+b^2\leq1$
    \item $a^2-b^2=\frac{1}{2}$
\end{enumerate}
\end{multicols}
\end{enumerate}
\end{document}