%iffalse
\documentclass[journal]{IEEEtran}
\usepackage[a5paper, margin=10mm, onecolumn]{geometry}
%\usepackage{lmodern} % Ensure lmodern is loaded for pdflatex
\usepackage{tfrupee} % Include tfrupee package

\setlength{\headheight}{1cm} % Set the height of the header box
\setlength{\headsep}{0mm}     % Set the distance between the header box and the top of the text

\usepackage{gvv-book}
\usepackage{gvv}
\usepackage{cite}
\usepackage{amsmath,amssymb,amsfonts,amsthm}
\usepackage{algorithmic}
\usepackage{graphicx}
\usepackage{textcomp}
\usepackage{xcolor}
\usepackage{txfonts}
\usepackage{listings}
\usepackage{enumitem}
\usepackage{mathtools}
\usepackage{gensymb}
\usepackage{comment}
\usepackage[breaklinks=true]{hyperref}
\usepackage{tkz-euclide} 
\usepackage{listings}
% \usepackage{gvv}                                        
\def\inputGnumericTable{}                                 
\usepackage[latin1]{inputenc}                                
\usepackage{color}                                            
\usepackage{array}                                            
\usepackage{longtable}                                       
\usepackage{calc}                                             
\usepackage{multirow}                                         
\usepackage{hhline}                                           
\usepackage{ifthen}                                           
\usepackage{lscape}
\begin{document}
\bibliographystyle{IEEEtran}
\title{JEE Main 2020 Paper - 4th September 2020 | Shift 1 (Maths)}
\author{EE24BTECH11016-Mokshith}
{\let\newpage\relax\maketitle}
\renewcommand{\thefigure}{\theenumi}
\renewcommand{\thetable}{\theenumi}
\setlength{\intextsep}{10pt} % Space between text and floats
\numberwithin{equation}{enumi}
\numberwithin{figure}{enumi}
\renewcommand{\thetable}{\theenumi}

\begin{enumerate}

\item Let $y=y\brak{x}$ be the solution of the differential equation, $xy^\prime-y=x^2\brak{x\cos x+\sin x},x>0$. If $y\brak{\pi}=\pi$, then $y'^\prime\brak{\frac{\pi}{2}}+y\brak{\frac{\pi}{2}}$ is equal to
    \begin{enumerate}
    \item $2+\frac{\pi}{2}+\frac{\pi^2}{4}$
    \item $2+\frac{\pi}{2}$
    \item $1+\frac{\pi}{2}$
    \item $1+\frac{\pi}{2}+\frac{\pi^2}{4}$
\end{enumerate}    
    \item The value of $\sum_{r=0}^{20}\comb{50-r}{6}$ is equal to:
    \begin{enumerate}
        \item $\comb{51}{7}-\comb{30}{7}$
        \item $\comb{51}{7}+\comb{30}{7}$
        \item $\comb{50}{7}-\comb{30}{7}$
        \item $\comb{50}{6}-\comb{30}{6}$
    \end{enumerate}
\item Let $\lfloor t\rfloor$ denote the greatest integer $\leq t$. Then the equation in $x$,$\lfloor x\rfloor^2+2\lfloor x+2\rfloor-7=0$ has:
\begin{enumerate}
    \item exactly four integral solutions
    \item infinitely many solutions
    \item no integral solution
    \item exactly two solutions
    \end{enumerate}
\item Let $\vec{P}\brak{3,3}$ be a point on the hyperbola, $\frac{x^2}{a^2}-\frac{y^2}{b^2}=1$. If the normal to it at $\vec{P}$ intersects the x-axis at $\brak{9,0}$ and $e$ is its eccentricity, then the ordered pair $\brak{a^2, e^2}$ is equal to:
\begin{enumerate}
    \item $\brak{9,3}$
    \item $\brak{\frac{9}{2},2}$
    \item $\brak{\frac{9}{2},3}$
    \item $\brak{\frac{3}{2},2}$
\end{enumerate}
\item Let $\frac{x^2}{a^2}+\frac{y^2}{b^2}=1$ $\brak{a>b}$ be a given ellipse, the length of whose latus rectum is 10. If its eccentricity is the maximum value of the function, $\phi\brak{t}=\frac{5}{12}+t-t^2$, then $a^2 + b^2$ is equal to
\begin{enumerate}
    \item 135
    \item 116
    \item 126
    \item 145
\end{enumerate}
\item Let $f\brak{x}=\int\frac{\sqrt{x}}{\brak{1+x}^2} dx$. Then $f\brak{3}-f\brak{1}$ is equal to:
\begin{enumerate}
    \item $-\frac{\pi}{6}+\frac{1}{2}+\frac{\sqrt{3}}{4}$
    \item $\frac{\pi}{6}+\frac{1}{2}-\frac{\sqrt{3}}{4}$
    \item $-\frac{\pi}{12}+\frac{1}{2}+\frac{\sqrt{3}}{4}$
    \item $\frac{\pi}{12}+\frac{1}{2}-\frac{\sqrt{3}}{4}$
\end{enumerate}
\item If $1+\brak{1-2^2\cdot1}+\brak{1-4^2\cdot3}+\brak{1-6^2\cdot5}+\cdots+\brak{1-20^2\cdot19}=\alpha-220\beta$, then an ordered pair $\brak{\alpha,\beta}$ is equal to:
\begin{enumerate}
    \item $\brak{10, 97}$
    \item $\brak{11, 103}$
    \item $\brak{11, 97}$
    \item $\brak{10,103}$
\end{enumerate}
\item The integral $\int\brak{\frac{x}{x\sin x+\cos x}}^2 dx$ is equal to (where $C$ is a constant of integration):
\begin{enumerate}
    \item $\tan x-\frac{x\sec x}{x\sin x+\cos x}+C$
    \item $\sec x+\frac{x\tan x}{x\sin x+\cos x}+C$
    \item $\sec x-\frac{x\tan x}{x\sin x+\cos x}+C$
    \item $\tan x+\frac{x\sec x}{x \sin x + \cos x}+C$
\end{enumerate}
\item Let $f\brak{x}=\abs{x-2}$ and $g\brak{x}=f\brak{f\brak{x}}$,$x\in \sbrak{0, 4}$. Then $\int_0^3 \brak{g\brak{x}-f\brak{x}} dx$ is equal to:
\begin{enumerate}
    \item $\frac{1}{2}$
    \item 0
    \item 1
    \item $\frac{3}{2}$
\end{enumerate}
\item Let $x_0$ be the point of local maxima of $f\brak{x}=\overrightarrow{a}\cdot\brak{\overrightarrow{b}\times \overrightarrow{c}}$, where $\overrightarrow{a}=x\hat{i}+2\hat{j}+3\hat{k}$,$\overrightarrow{b}=-2\hat{i}+x\hat{j}-\hat{k}$, and $\overrightarrow{c}=7\hat{i}-2\hat{j}+x\hat{k}$. Then the value of $ \overrightarrow{a}\cdot \overrightarrow{b}+\overrightarrow{b}\cdot \overrightarrow{c}+\overrightarrow{c}\cdot \overrightarrow{a}$ at $x=x_0$ is:
\begin{enumerate}
    \item -22
    \item -4
    \item -30
    \item 14
\end{enumerate}
\item A triangle $ABC$ lying in the first quadrant has two vertices as $ \vec{A}\brak{1,2}$ and $\vec{B}\brak{3,1}$. If $\angle BAC=90^\circ$, and ar$\brak{\triangle ABC}$ is $5\sqrt{5}$ s units, then the abscissa of the vertex $\vec{C}$ is:
\begin{enumerate}
    \item $1+\sqrt{5}$
    \item $1+2\sqrt{5}$
    \item $2\sqrt{5}-1$
    \item $2+\sqrt{5}$
\end{enumerate}
\item Let $f$ be a twice differentiable function on $\brak{1,6}$. If $f\brak{2}=8, f^\prime\brak{2}=5, f^\prime\brak{x}\geq1$ and $f'^\prime\brak{x}\geq4$, for all $x\in \sbrak{1,6}$, then:
\begin{enumerate}
    \item $f\brak{5}+f^\prime\brak{5}\geq28$
    \item $f^\prime\brak{5}+f'^\prime\brak{5}\leq20$
    \item $f\brak{5}\leq10$
    \item $f\brak{5}+f^\prime\brak{5}\leq26$
\end{enumerate}
\item Let $\alpha$ and $\beta$ be the roots of $x^2-3x+p=0$ and $\gamma$ and $\delta$ be the roots of $x^2-6x+q=0$. If $\alpha,\beta,\gamma,\delta$ form a geometric progression. Then ratio $\brak{2q+p}:\brak{2q-p}$ is:
\begin{enumerate}
    \item 33 :31
    \item 9 :7
    \item 3 :1
    \item 5 :3
\end{enumerate}
\item Let $u=\frac{2z+i}{z-ki},z=x+iy$ and $k>0$. If the curve represented by $Re\brak{u}+Im\brak{u}=1$ intersects the y-axis at the points $\vec{P}$ and $\vec{Q}$ where $PQ=5$, then the value of $k$ is:
\begin{enumerate}
    \item 4
    \item $\frac{1}{2}$
    \item 2
    \item $\frac{3}{2}$
\end{enumerate}
\item If $A=\myvec{\cos\theta&i\sin\theta\\i\sin\theta&\cos\theta},\left(\theta=\frac{\pi}{24}\right)$ and $A^5=\myvec{a&b\\c&d}$,where $i=\sqrt{-1}$,then which one of the following is not true?
\begin{enumerate}
    \item $a^2-d^2=0$
    \item $a^2-c^2=1$
    \item $0\leq a^2+b^2\leq1$
    \item $a^2-b^2=\frac{1}{2}$
\end{enumerate}

\end{enumerate}
\end{document}